
%Gravimetrie-Versuchsbeschreibung

Bei der Gravimetrie-Messung wurden drei verschiedene Messmethoden angewandt. Das sind die Messung mit dem Gravimeter, Vermessung von Punkten mit der Tachymetrie und GPS.
Unsere Fragestellung ist hier in erster Linie wie gut der Basaltgang mit der Gravimetrie im Vergleich zu den anderen Messverfahren untersucht werden kann.  Wir wollen die Lage des Ganges möglicht genau bestimmen, und mit den 
Ergebnissen der übrigen Messmethoden vergleichen.

\section{Messung mit dem Gravimeter}
Die Messung mit den beiden vorhandenen Gravimetern wurde auf dem Profil orthogonal zum Basaltgang durchgeführt. Auf dem gleichen Profil wurden schon Messungen mit allen drei der vorherigen Messverfahren durchgeführt.
Das Messprofil ist in Abb. ??? zu sehen. Da zwei Gravimeter zur Verfügung standen, wurde mit jedem der Messgeräte jeweils jeder zweite Punkt Vermessen. Der Messpunkt $G0$ in Abb.??? bezeichnet den Basispunkt der Messung, auf
diesem Punkt wurde mit beiden Gravimetern eine Messung durchgeführt. Bei $G7$ wird das Maximum des Basaltgangs erwartet. Deswegen wurden dort die Messabstände am kleinsten gewählt.\\
Um den Drift der Messgeräte erfassen und korrigieren zu können wurden jeweils zwei Messungen auf dem selben Profil im Abstand von mehr als einer Stunde durchgeführt. Dies wurde umgesetzt indem man den gen Gravimetern 
einmal das ganze Profil komplett entlang ging und dann mit der zweiten Messung wieder am ersten Messpunkt begonnen hat.\\  
\\

Die Messungen wurden mit Gravimetern des Types LaCoste-Romberg(G) durchgeführt. Da Gravimeter sehr empfindliche Messgeräte sind mussten die Messungen sehr genau und vorsichtig durchgeführt werden. Das Messgerät hat eine 
theoretische Auflösung von 0,01 mGal.\\
\\
Um einen Messpunkt zu vermessen wurde das Gravimeter zunächst neben den mit zwei Pflöcken markierten Punkt gestellt. Mit Hilfe zweier Libellen wurde es horizontal ausgerichtet. Einer der beiden Pflöcke 
war bis auf Handbreite in den Boden geschlagen und diente zur exakten Messung der Instrumentenhöhe. Diese Höhe sollte bis auf 5 mm genau bestimmt werden.\\
Wenn das Gravimeter nicht mehr bewegt wurde kann die Arretierung gelöst werden. Solange die Messung durchgeführt wurde, achtete man darauf das sich keine Person dem Messgerät nähert oder sich die Personen in der 
Nähe stark bewegen. Dies war wichtig da es einen sofort sichtbaren Einfluss auf die Messung hatte, wenn sich eine Person genähert hat.\\
Die Messung wurde immer zu dritt durchgeführt. Eine Person war verantwortlich für einen Sonnenschirm, der über das Messgerät gehalten wurde, eine für das Ablesen der Messwerte 
und eine dritte Person hat das Messprotokoll geführt. Um Die Gezeitenkorrektur berechnen zu können wurde auch direkt die Zeit aufgeschrieben.\\


\section{Tachymetrie}???
Mit einem Tachymeter wurde die Lage der Messpunkte, die mit den Gravimetern vermessen wurden, bestimmt. Dazu wird der Reflektor an dem zu vermessenden Punkt aufgestellt und mit dem Messgerät angepeilt. Mit einem Laser 
wird die Entfernung gemessen.

\section{Höhenmessung mit GPS}









Die Notizen, die während der Messungen gemacht wurden, befinden sich im Anhang in den Abbildungen \ref{fig:mitschrieb1} und \ref{fig:mitschrieb2}

\section{Messung mit dem Gravimeter}

Die Messung mit den beiden vorhandenen Gravimetern wurde auf dem Profil orthogonal zum Basaltgang durchgeführt. Auf dem gleichen Profil wurden schon Messungen mit allen drei der vorherigen Messverfahren durchgeführt.
Das Messprofil ist in Abb. ??? zu sehen. Da zwei Gravimeter zur Verfügung standen, wurde mit jedem der Messgeräte jeweils jeder zweite Punkt vermessen. Der Messpunkt $G0$ in Abb.??? bezeichnet den Basispunkt der Messung, auf
diesem Punkt wurde mit beiden Gravimetern eine Messung durchgeführt. Bei $G7$ wird das Maximum des Basaltgangs erwartet. Deswegen wurden dort die Messabstände am kleinsten gewählt.

Um die Drift der Messgeräte erfassen und korrigieren zu können wurden jeweils zwei Messungen auf dem selben Profil im Abstand von mehr als einer Stunde durchgeführt. Dies wurde umgesetzt indem man mit den Gravimetern 
einmal das ganze Profil komplett entlang ging und dann mit der zweiten Messung wieder am ersten Messpunkt begonnen hat.

Die Messungen wurden mit Gravimetern des Types LaCoste-Romberg(G) durchgeführt. Da Gravimeter sehr empfindliche Messgeräte sind mussten die Messungen sehr genau und vorsichtig durchgeführt werden. Das Messgerät hat eine 
theoretische Auflösung von 0,01\,mGal.

Um einen Messpunkt zu vermessen wurde das Gravimeter zunächst neben den mit zwei Pflöcken markierten Punkt gestellt. Mit Hilfe zweier Libellen wurde es horizontal ausgerichtet. Einer der beiden Pflöcke 
war bis auf Handbreite in den Boden geschlagen und diente zur exakten Messung der Instrumentenhöhe. Diese Höhe sollte bis auf 5\,mm genau bestimmt werden.
Wenn das Gravimeter nicht mehr bewegt wurde, konnte die Arretierung gelöst werden. Solange die Messung durchgeführt wurde, achtete man darauf das sich keine Person dem Messgerät näherte oder sich die Personen in der 
Nähe stark bewegen. Dies war wichtig da es einen sofort sichtbaren Einfluss auf die Messung hatte, wenn sich eine Person genähert hat.

Die Messung wurde immer zu dritt durchgeführt. Eine Person war verantwortlich für einen Sonnenschirm, der über das Messgerät gehalten wurde, eine für das Ablesen der Messwerte 
und eine dritte Person hat das Messprotokoll geführt. Im Messprotokoll (siehe Abbildungen \ref{fig:MP1_156} bis \ref{fig:MP2_686} im Anhang) wurde die Instrumentenhöhe und die Zeit direkt mit aufgeschrieben, um noch vor Ort die Instrumentenhöhenkorrektur nach Gleichung \eqref{eq:Freiluft} und die Gezeitenkorrektur mit einem Diagramm durchzuführen. Dieses befindet sich im Anhang in Abbildung ???. Es ist hilfreich, diese Korrekturen direkt vor Ort durchzuführen, um die Messungen der verschiedenen Geräte, die zu unterschiedlichen Zeiten durchgeführt wurden, vergleichen zu können und schon im Feld feststellen zu können, ob die Daten auswertbar werden oder es größere Messunsicherheiten gab.

\section{Tachymetrie}???
Mit einem elektronischen Tachymeter TC 500 wurde die Lage der Messpunkte, die mit den Gravimetern vermessen wurden, bestimmt. Dazu wurde der Reflektor an dem zu vermessenden Punkt aufgestellt und mit dem Messgerät angepeilt. Mit einem Laser wurde die Entfernung gemessen. Die Messprotokolle zu diesem Versuchsteil sind im Abhang unter Abbildung \ref{fig:MPTachymetrie1} und \ref{fig:MPTachymetrie2} zu finden.

\section{Vermessung mit GPS}

Mit einem GPS-Gerät wurden die Pflöcke, mit denen alle Profil-Punkte und sonstigen wichtigen Orte während der Messungen markiert wurden, eingemessen. Es wurden die GPS-Koordinaten, die Zeit der Messung und die Art der Lösung des Gleichungssystems gespeichert. Manchmal hatte das Messgerät durch zum Beispiel Bäume keinen Kontakt zu genug Satelliten, sodass die Lage der Punkte nicht so genau wie sonst bestimmt werden konnte.

Bei der Messung selbst musste darauf geachtet werden, dass die Stange an der das Gerät befestigt war, mit zwei Stangen möglichst genau vertikal ausgerichtet wurde. Zur Kontrolle der Ausrichtung diente eine Punkt-Libelle. Außerdem musste ihre Länge notiert werden, um auf die Koordinaten auf Boden-Niveau schließen zu können. Nach einer Messung konnten der vermessene Punkt im Gerät umbenannt werden, um die Übersichtlichkeit zu gewährleisten. Hier machte sich bemerkbar, wie hilfreich unsere während der Versuchstage angefertigte Karte der Profilpunkte war. Außerdem stellte sich unsere Beschriftung der Pflöcke als sehr hilfreich und strukturiert heraus.

