Mit der Gravimetrie-Messung ließ sich der Basaltgang beobachten und auch Modelle zu dessen Ausdehnung und Lage finden, wovon zwei in diesem Protokoll vorgestellt und beschrieben wurden. Diese Modelle werden noch mit den anderen Messverfahren verglichen.

Die Breite des Gangs konnte mit der Gravimetrie, je nach Modell, als $\e{(4,41\pm0,40)}{m}$ bzw. $\e{(4,54\pm0,58)}{m}$ bestimmt werden. Mit der Magnetik wurde der Gang entlang des gleichen Profils untersucht. Es ergab sich eine Breite von $\e{(4,4\pm0,5)}{m}$, was sehr gut mit den Gravimetrie-Breiten übereinstimmt. Der Unterschied der Modelle ist jedoch, dass der Neigungswinkel bei der Magnetik als $(16\pm3)^\circ$ bestimmt werden konnte und in der Gravimetrie je nach Modell $30^\circ$ bzw. $0^\circ$ war. Auch die Tiefen $\e{(0,397\pm0,010)}{m}$ bzw. 1,5\,m der Gravimetrie-Messungen weichen von der der Magnetik $\e{(1,6\pm0,5)}{m}$ ab.

Mit der Seismik wurde auch versucht, die Tiefe der Gangoberkante zu bestimmen. Dies gelang jedoch nicht, weswegen die Ergebnisse nur soweit verglichen werden können, dass sich die Gravimetrie viel besser, also überhaupt, zur Untersuchung des Basaltgangs eignet.

Mit der Geoelektrik konnte der Basaltgang auch über dem gleichen Profil beobachtet werden, jedoch eignete sich dieses Verfahren nicht dazu, Zahlenwerte zur Ausdehnung oder Tiefenlage des Basaltgangs zu bestimmen. Die Ergebnisse können also nicht weiter verglichen werden.

Zusammenfassend lässt sich also sagen, dass die Gravimetrie-Messung durch die Sorgfalt, mit der sie hier durchgeführt wurde, in der Lage ist, sehr gute Modelle für den Basaltgang zu finden. Dennoch sind diese natürlich nicht als eindeutig richtig zu bewerten, weil das generelle Problem der Mehrdeutigkeit in der Geophysik nicht beseitigt werden kann. 