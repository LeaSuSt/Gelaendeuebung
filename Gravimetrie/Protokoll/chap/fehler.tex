Da der Messwert und die Korrekturwerte pro Messpunkt nur ein- bis zweimal erfasst werden, kann keine sinnvolle statistische Aussage über die Messfehler getroffen werden. Stattdessen kann ein Größtfehler abgeschätzt werden, der aus den einzelnen Ableseungenauigkeiten der Messwerte mit Größtfehlerabschätzung berechnet werden kann. Dazu muss man sich überlegen, wie genau die einzelnen Messgrößen abgelesen werden konnten und wie genau die Messgerät positioniert werden konnten.

Als Unsicherheit auf die Messung der Gerätehöhe mit dem Zollstock nehmen wir eine Ungenauigkeit von
\begin{equation}
 \q{delta}{\q{x}{Zollstock}}=\e{0,005}{m}
\end{equation}
an. Dabei wurde berücksichtigt, dass die Ableseungenauigkeit bei einer halben Skaleneinheit liegt, der Zollstock eventuell nicht ganz vertikal gehalten wurde und auch die Wasserwaage möglicherweise nicht ganz genau horizontiert wurde. Der Fehler auf das Messrad wird mit
\begin{equation}
 \q{delta}{\q{x}{Messrad}}=\e{0,005}{Skt}
\end{equation}
angenommen, weil die dritte Nachkommastelle zwischen zwei Strichen abgeschätzt werden musste. Dies entspricht also einer halben Skaleneinheit der letzten genau abzulesenden Stelle. Der Fehler auf die Ablesung aus der Gezeitentabelle wird als
\begin{equation}
 \q{delta}{\q{x}{Gezeiten}}=\e{0,002}{mGal}
\end{equation}
angenommen und der Fehler auf die Genauigkeiten der Horizontierung  mit den Libellen ist
\begin{equation}
 \q{delta}{\q{x}{Libelle}}=0,725\cdot 10^{-4}
\end{equation}
