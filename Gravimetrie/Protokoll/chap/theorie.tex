\section{Schwerebeschleunigung}

Die Gravitationsbeschleunigung aufgrund der Anziehung zweier Massen ist
\begin{equation}
 \vec{b}_a(\vec{r})=G\int_V \frac{\rho(\vec{r}')}{|\vec{r}'-\vec{r}|^3}(\vec{r}'-\vec{r})\text{d}V'
\end{equation}
mit der Gravitationskonstanten
\begin{equation}
 G=\eb{6,673\cdot 10^{-11}}{m^3}{kg\,s^2} \fullstop
\end{equation}

Außerdem wirken noch scheinbare Schwerewirkungen wie zum Beispiel die Zentrifugalbeschleunigung $\vec{z}(\vec{r})$, die durch die Erdrotation hervorgerufen wird.

Wird eine Messung der Schwere durchgeführt, wird immer die vektorielle Summe dieser beiden Beschleunigungen
\begin{equation}
 \vec{g}_a(\vec{r})=\vec{b}_a(\vec{r})+\vec{z}(\vec{r})
\end{equation}
gemessen.

Die SI-Einheit der Schwerebeschleunigung ist $\left[g\right]=\eb{1}{m}{s^2}$, wobei in der Geophysik meist die Einheit $\e{1}{Gal}$ verwendet wird. Die Umrechnung erfolgt durch
\begin{equation}
 \eb{1}{m}{s^2}=\e{100}{Gal} \fullstop
\end{equation}

\section{Reduktionen}

Meist möchte man durch die gemessenen Schwereunterschiede auf einen unbekannten Dichteunterschied im Untergrund schließen. Dazu müssen jedoch die bekannten Schwereunterschiede aus den Messwerten eliminiert werden. Es wird von Reduktion der Schwerewerte gesprochen.

Die Freiluftreduktion (oder auch Niveaureduktion) berücksichtigt, dass die Schwere mit zunehmender Höhe $H$ abnimmt. Es gilt dabei
\begin{equation}
 \q{\delta}{g\,Freiluft}=\eb{-0,3086\cdot H}{mGal}{m} \fullstop
\end{equation}
Den gleichen Effekt muss man bei der Instrumentenhöhe beachten, was zu $\q{\delta}{g\,Instrumentenhöhe}$ führt. Die Instrumentenhöhe wird jedoch direkt bei der Messung aus den Messwerten eliminiert, weil dies eine von der Geometrie der Messung unabhängige Auswertung ermöglicht.

Bei der Gezeitenreduktion $\q{\delta}{g\,Gezeiten}$ wird die Gravitationswirkung anderer Planeten berücksichtigt, die sich mit der Zeit verändert.

% Drift???

Die Geländereduktion $\q{\delta}{g\,Gelände}$ berücksichtigt die Verringerung des Schwerewerts durch zum Beispiel Berge und Täler in der Nähe des Messorts. Nach der Korrektur ist der Messwert also vergrößert.

Die Bouguerreduktion berücksichtigt den Effekt, dass eine Gesteinsschicht zwischen Referenzfläche und Messpunkt den gemessenen Wert der Schwere vergrößert. Es wird angenommen, dass diese Gesteinsschicht aus homogenem Material der Dichte $\rho$ besteht. Dies führt zur Gleichung
\begin{equation}
 \q{\delta}{g\,Bouguer}=2\pi\rho GH=\eb{0,0419\cdot H\cdot\rho\cdot 10^{-8}}{m^3}{s^2kg}
\end{equation}
für diese Reduktion.

Die Breitenreduktion beinhaltet die räumlichen Schwereunterschiede aufgrund der breitenabhängigen Zentrifugalbeschleunigung, die durch die Erdrotation hervorgerufen wird. Der Schwerewert vergrößert sich hierbei mit zunehmender Breite. Die Normalschwereformel
\begin{equation}
 \q{\delta}{g\,Breite}=\eb{8,2\cdot L\cdot 10^{-9}}{1}{s^2}
\end{equation}
liefert mit dem Abstand $L$ zur geographischen Breite des Referenzpunkts den Korrekturwert.

% regionale Reduktion???

Von einer Bougueranomalie $\q{g}{Bouguer}$ wird gesprochen, wenn all diese Reduktionen auf den Messwert $\q{g}{mess}$ angewendet wurden. Es gilt also
% regio??? Drift???
\begin{equation}
 \q{g}{Bouguer}=\q{g}{mess}-\q{\delta}{g\,Instrumentenhöhe}-\q{\delta}{g\,Gezeiten}-\q{\delta}{g\,Gelände}-\q{\delta}{g\,Freiluft}-\q{\delta}{g\,Breite}-\q{\delta}{g\,Bouguer}
\end{equation}

\section{Dichtebestimmung über bekannte Störkörper???}

% Was machen wir dazu???

\section{Gravimeter}



\section{Bestimmung der Höhe???}

% Nivellement??? -> wird nicht verwendet, Tachymeter??? GPS???