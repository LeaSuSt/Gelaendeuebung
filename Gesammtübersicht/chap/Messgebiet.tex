%Geammtübersicht Messgebiet

\section{Das Hegau}
Das Hegau ist eine Gebiet dass grob zwischen dem Bodensee und der schwäbischen Alb liegt(Wikipedia)???. Charakteristisch für dieses Gebiet sind die vulkanisch 
geprägten Hegauer Kegelberge. Diese Kegelberge sind Schlothe erlöschener Vulkane. Der Vulkanismus des Hegaugebiets hat seinen Ursprung in der Mitte des Miozän, 
was vor etwa 14 Millionen Jahren war. Es entstanden duzende Vulkane, in deren Schlothe vor ca. 9 Millionen Jahren auch der Hegauer Basalt erstarrte.\\
Im Pleistozän gab es eine Eiszeit in diesem Gebiet. Durch die entstandenen Kletscher wurde Molasse und Tuff abgetragen, es blieben, die heute noch zu sehenden, 
Phonolithkerne und Basaltkerne stehen. 






\section{Die Messgebiete}


Es gab im Wesentlichen zwei Messgebiete auf denen wir unsere Messungen durchgeführt haben, sie liegen über Riedheim. In Abbildung \ref{abb:Messgebiete} sind diese Messgebiete 
eingezeichnet. Unter Messgebiet 1 liegt der Basaltgang, hier wurde mit allen vier Messmethoden Messungen durchgeführt. 
Auf Messgebiet 2 wurde nur mit Geoelektrik und Seismik gemessen.
gemessen.
\begin{figure}
 \centering
 \includegraphics[width=0.9\textwidth]{fig/Vergleich_nahe_Profile.pdf}
 \caption[Messgebiete]{Messgebiete auf denen unsere Messungen durchgeführt wurden. Unter Messgebiet 1 ist der Basaltgang.}
 \label{abb:Messgebiete}
\end{figure}

Aberhalb den Messgebiets 1 ist eine Steinbruch in dem Basalt frei gelegt ist. Der Aufschluss dieses Basaltgangs ist etwa 