Insgesamt bewerten wir die aus den Magnetik-Messungen gewonnenen Ergebnisse als sehr gut. Dies liegt besonders daran, dass auf den Ergebnissen der Magnetik-Kartierung die gesamten weiteren Planungen der Messungen während der Messwoche aufbauten. Dadurch lagen unsere Profile senkrecht zum Gang, was für die Magnetik-Auswertung von großer Bedeutung war. Für die Gravimetrie reichte es, den Winkel zur Streichrichtung des Gang zu kennen, was durch die senkrechte Lage auch gegeben war.

Die Fragestellung nach der Lokalisierung des Gangs kann bejaht werden, da dieser mit der Kartierung eindeutig lokalisiert werden konnte. Die Lokalisierung reichte auch aus, um die Lage der Profile festzulegen. Auch das extra schräg zum Gang angeordnete Profil erfüllte seinen Zweck und zeigte uns, dass eine schräge Lage zum Gang einen sehr viel breiteren Verlauf des Maximums zur Folge hat.

Für die drei nah aneinander liegenden Profile konnte ein Modell gefunden werden, das den Verlauf der Totalintensität über allen drei Profilen erklärt. Die Richtigkeit dieses Modell ist jedoch nur schwer zu überprüfen, weil bei der Modellierung viele Annahmen getroffen werden mussten. Es ergibt sich jedoch bei Profil M21-M2 eine Breite von $\e{(4,4\pm0,5)}{m}$ bei einem Neigungswinkel von $(16\pm3)^\circ$ gemessen von der senkrechten aus in Richtung Ost. Die Tiefe der Gangoberkante bei diesem Profil als $\e{(1,6\pm0,5)}{m}$ bestimmt werden.

Diese Ergebnisse können gut mit denen der Gravimetrie verglichen werden, weil die Gravimetrie-Messung auch entlang des Profils M21-M2 durchgeführt wurde. Es ergab sich, je nach Modell, eine Breite von $\e{(4,41\pm0,40)}{m}$ bzw. $\e{(4,54\pm0,58)}{m}$. Alle drei Breiten liegen in den Fehlergrenzen der jeweils anderen Messungen. Die Vergleichbarkeit ist dennoch nicht wirklich gut gegeben, weil die Tiefe der Gangoberkante und der Neigungswinkel bei den Gravimetrie-Modellierungen ganz andere Werte ergaben. Die Tiefe wurde als $\e{(0,397\pm0,010)}{m}$ bzw. 1,5\,m modelliert und der Neigungswinkel war $30^\circ$ bzw. $0^\circ$.

Die Fluxgate-Messungen zeigten, dass das Zweikreisverfahren für unsere Messungen nicht angewendet werden kann und die Vermessung eines Traktors und der Umgebung der Basisstation halfen bei der Einschätzung der während der Messungen gemachten Fehler.