%Gravimetrie-Versuchsbeschreibung

Bei der Gravimetrie-Messung wurden drei verschiedene Messmethoden angewandt. Das sind die Messung mit dem Gravimeter, Vermessung von Punkten mit der Tachymetrie und GPS.
Unsere Fragestellung ist hier in erster Linie wie gut der Basaltgang mit der Gravimetrie im Vergleich zu den anderen Messverfahren untersucht werden kann.  Wir wollen die Lage des Ganges möglicht genau bestimmen, und mit den 
Ergebnissen der übrigen Messmethoden vergleichen.

\section{Messung mit dem Gravimeter}
Die Messung mit den beiden vorhandenen Gravimetern wurde auf dem Profil orthogonal zum Basaltgang durchgeführt. Auf dem gleichen Profil wurden schon Messungen mit allen drei der vorherigen Messverfahren durchgeführt.
Das Messprofil ist in Abb. ??? zu sehen. Da zwei Gravimeter zur Verfügung standen, wurde mit jedem der Messgeräte jeweils jeder zweite Punkt Vermessen. Der Messpunkt $G0$ in Abb.??? bezeichnet den Basispunkt der Messung, auf
diesem Punkt wurde mit beiden Gravimetern eine Messung durchgeführt. Bei $G7$ wird das Maximum des Basaltgangs erwartet. Deswegen wurden dort die Messabstände am kleinsten gewählt.\\
Um den Drift der Messgeräte erfassen und korrigieren zu können wurden jeweils zwei Messungen auf dem selben Profil im Abstand von mehr als einer Stunde durchgeführt. Dies wurde umgesetzt indem man den gen Gravimetern 
einmal das ganze Profil komplett entlang ging und dann mit der zweiten Messung wieder am ersten Messpunkt begonnen hat.\\  
\\

Die Messungen wurden mit Gravimetern des Types LaCoste-Romberg(G) durchgeführt. Da Gravimeter sehr empfindliche Messgeräte sind mussten die Messungen sehr genau und vorsichtig durchgeführt werden. Das Messgerät hat eine 
theoretische Auflösung von 0,01 mGal.\\
\\
Um einen Messpunkt zu vermessen wurde das Gravimeter zunächst neben den mit zwei Pflöcken markierten Punkt gestellt. Mit Hilfe zweier Libellen wurde es horizontal ausgerichtet. Einer der beiden Pflöcke 
war bis auf Handbreite in den Boden geschlagen und diente zur exakten Messung der Instrumentenhöhe. Diese Höhe sollte bis auf 5 mm genau bestimmt werden.\\
Wenn das Gravimeter nicht mehr bewegt wurde kann die Arretierung gelöst werden. Solange die Messung durchgeführt wurde, achtete man darauf das sich keine Person dem Messgerät nähert oder sich die Personen in der 
Nähe stark bewegen. Dies war wichtig da es einen sofort sichtbaren Einfluss auf die Messung hatte, wenn sich eine Person genähert hat.\\
Die Messung wurde immer zu dritt durchgeführt. Eine Person war verantwortlich für einen Sonnenschirm, der über das Messgerät gehalten wurde, eine für das Ablesen der Messwerte 
und eine dritte Person hat das Messprotokoll geführt. Um Die Gezeitenkorrektur berechnen zu können wurde auch direkt die Zeit aufgeschrieben.\\


\section{Tachymetrie}???
Mit einem Tachymeter wurde die Lage der Messpunkte, die mit den Gravimetern vermessen wurden, bestimmt. Dazu wird der Reflektor an dem zu vermessenden Punkt aufgestellt und mit dem Messgerät angepeilt. Mit einem Laser 
wird die Entfernung gemessen.

\section{Höhenmessung mit GPS}










