\section{Basisstation}

Zuerst wurde die Basisstation aufgebaut. Sie besteht aus einem Torsions-Magneto"|meter, das die Vertikalkomponente des Magnetfelds misst, und steht während des gesamten Versuchstags am gleichen Ort. Dadurch können zeitliche Variationen während der Versuche erkannt werden und die Messergebnisse eventuell um diese Variationen korrigiert werden. Würde eine solche Messung nicht durchgeführt werden, könnten zeitliche Variationen im Erdmagnetfeld fälschlicherweise als örtliche Anomalien interpretiert werden. Beim Aufbau des Torsions-Magnetometer musste besonders auf die genaue Horizontierung geachtet werden, damit auch wirklich nur die Horizontalkomponente des Erdmagnetfelds gemessen wurde.

Zuerst wurde das Torsions-Magnetometer kalibriert. Dazu wurde mit Hilfe eines Helmholtzspulenpaars ein über den Strom bekanntes Magnetfeld am Ort des Torsions-Magnetometers erzeugt. Durch eine lineare Regression kann in der Auswertung dann der Kalibrierungsfaktor $\frac{\tau}{|\vec{m}|}$ bestimmt werden.

Im Laufe des Tages wurde dann von 12:35 Uhr bis 17:17 Uhr alle 10 bis 20 Minuten ein Messwert an der Basisstation aufgenommen und Besonderheiten bei der Messung notiert. Beispielsweise war hin und wieder ein Traktor n der Nähe, von dem nicht klar war, ob er auf diese Entfernung schon  die Messung beeinflussen könnte. Das Torsions-Magnetometer musste auch einmal neu horizontiert werden.

\section{Kartierung}

Als nächstes wurde eine Kartierung mit einem Gradiometer durchgeführt. Um ein geeignetes Quadrat zur Durchführung dieser zu finden, wurde die Messwiese mit dem Gradiometer abgelaufen und Stellen mit großem Gradienten gesucht. So hatten wir bereits eine Idee, wo der Basaltgang verlaufen könnte. Außerdem entnahmen wir der geologischen Karte, dass sich der Basaltgang ungefähr in Nord-Süd-Richtung vom Steinbruch, an dem der Gang an der Oberfläche aufgeschlossen ist, aus erstreckt. In der Karte ist dieser als magnetische Anomalie eingezeichnet

Wir wählten dann das Quadrat M1-M2-M3-M4 mit einer Kantenlänge von 30 Metern (Lage siehe Abbildung ???), um die Kartierung durchzuführen. Dazu wurden Maßbänder entlang der Kanten möglichst senkrecht zueinander ausgelegt und dann mit möglichst gleichmäßiger Geschwindigkeit innerhalb von 30 Sekunden die Kante M1-M2 abgelaufen. Die Samplingrate in Laufrichtung betrug 8 Messungen pro Sekunde, also bei der richtigen Laufgeschwindigkeit auch 8 Messungen pro Meter. Dies wurde in 1\,Meter-Abständen 30 mal wiederholt. Es ergibt sich also eine Kartierung auf diesem Quadrat aus insgesamt 720 Messpunkten.

% Zu Fehlerbetrachtungen????
Dabei auftretende Schwierigkeiten waren, dass manchmal ein Busch im Laufweg war und um diesen herumgelaufen werden musste. Außerdem fuhr einmal während der Messung ein Traktor vorbei. Eine weitere Schwierigkeit stellte die erforderliche gleichmäßige Geschwindigkeit dar. Nach der Hälfte wechselte die Beobachterin. Die zweite Beobachterin hatte nur einen Testlauf und auch bei den nächsten drei Bahnen kam sie manchmal ein bis zwei Sekunden zu früh oder spät am Ende des Quadrats an.

\section{Profile}

Nun musste bestimmt werden, entlang welcher Profile Messungen durchgeführt werden sollten. Dabei war entscheidend, dass die Software, mit welcher wir diese Messungen auswerten, von einer senkrechten Lage des Profils zum zu untersuchenden Objekt im Untergrund ausgeht.  Wir konnten bereits vor Ort das Ergebnis der Kartierung anschauen und wussten so, dass der Basaltgang parallel zur Diagonalen M1-M3 verläuft. Um dazu senkrecht zu messen, legten wir unsere Profile entlang der Diagonalen M4-M2. Diese Profile nannten wir M22-M23, M21-M2 und M24-M25 (Lage siehe Abbildung ???). Sie liegen im Abstand von drei Metern nebeneinander. Das Profil M26-M27 ist auch senkrecht zum Gang angeordnet, aber aus dem Kartierungsergebnis zu schließen vermutlich nicht ganz mittig darüber. Um zu prüfen, ob in den Messergebnissen überhaupt zu erkennen ist, ob senkrecht oder in einem anderen Winkel zum Gang gemessen wurde, ist das Profil M28-M29 absichtlich schräg zum Gang angelegt worden. Ein weiteres Profil ohne Name verlief von der Basisstation aus am Tisch und der Hütte vorbei, um den Einfluss der Umgebung auf das Magnetfeld an der Basisstation zu untersuchen.

Alle Profile wurden mit einem der drei Protonen-Präzessions-Magnetometern 1, 2 und neu ???? vermessen. Die Profile M21-M2 und M24-M25 wurden außerdem mit einem Fluxgate-Magnetometer vermessen. Diese Messung diente jedoch nur dazu, dass wir dieses Messgerät auch kennenlernten und um einen Vergleich der Genauigkeit der verschiedenen Messgeräte zu erhalten.

\section{Vergleich der Messgeräte}

Am Ende des Versuchstags wurden mit jedem Protonen-Präzessions-Magnetometer drei Messungen am Ort der Basisstation durchgeführt. Auch mit dem Fluxgate-Magnetometer wurde eine Messung an der Basisstation durchgeführt. Diese Messung diente zur Überprüfung, ob die verschiedenen Magnetometer den gleichen Wert anzeigen. Falls dies nicht der Fall sein sollte, können so dennoch die Messwerte verglichen und so korrigiert werden, dass sie untereinander vergleichbar sind.