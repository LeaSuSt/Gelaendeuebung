Bei der geophysikalischen Geländeübung 2018 führten wir am ersten Messtag, den 22.05., die Magnetik-Messungen durch. Das folgende Protokoll beschreibt die Theorie, Versuchsdurchführung, Auswertung und Fehlerdiskussion zu diesem Versuch.

Bei den Messungen und im Protokoll verfolgten wir folgende Fragestellungen: Die erste große Fragestellung ist, ob mit der Magnetik der Gang lokalisiert werden kann. Bei der Kartierung stellten wir uns die Frage, ob wir den Gang so gut lokalisieren können, dass wir uns daran die Lage der weiteren Profile überlegen können.
Bei den Profilen stellten wir uns dann die Frage, ob diese wirklich senkrecht zum Gang angelegt wurden. Dazu dient vor allem ein Profil, das extra schräg zum Gang gewählt wurde, um zu sehen, ob überhaupt ein Unterschied festgestellt werden kann. Die senkrechten Profile sollen zeigen, ob ein gemeinsames, alle Verläufe der Totalintensitäten erklärendes Modell gefunden werden kann. Die Messungen mit dem Fluxgate verfolgen die Fragestellung, ob dabei sinnvoll das Zweikreisverfahren angewendet werden kann. Die Vermessung eines vorbeifahrenden Traktors und der Umgebung der Basisstation dienten zur Abschätzung der während der Messung durch äußere Einflüsse auftretenden Fehler.