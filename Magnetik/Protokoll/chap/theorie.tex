\section{Einführung}

Mit diesem Verfahren kann die Magnetisierung und magnetische Suszebilität von Objekten im Untergrund untersucht werden, sowie Variationen des Erdmagnetfelds. Es kann sowohl die räumliche als auch die zeitliche Variation 
betrachtet werden.

Das gesamte gemessene Magnetfeld setzt sich aus dem Hintergrundfeld und Anomalien zusammen. Diese Anomalien 
 werden durch magnetisierbare und remanent magnetisierbare Gesteine hervorgerufen. 

Die Messgröße die in diesem Versuch gemessen wird ist die \textbf{magnetische Flussdichte} $\vec{B}$, sie gibt die Stärke des Magnetfeldes an.

Eine andere wichtige Größe ist \textbf{Magnetisierung} $\vec{M}$ eines Materials, sie hängt sowohl von der Stärke des Magnetfelds als auch den Materialeigenschaften ab.\\

\section{Induzierte und remantente Magnetisierung}
Gesteine oder sonstige Störkörper können remanent oder induziert magnetisiert sein. Induzierte Magnetisierung ist Magnetisierung, die nur solange ein äußeres Magnetfeld angelegt ist, vorhanden ist. Remanent magnetisierte 
Materialien sind dauerhaft magnetisch, auch ohne äußeres Magnetfeld.\\
Dia- und paramagnetische Materialien weisen nur induzierte Magnetisierung auf. Ferro-, ferri- und antiferromagnetische Materialien können auch remanent magnetisiert sein. \\
Daneben gibt es noch die thermoremanente Magnetisierung, hierbei entsteht die Magnetisierung während Gesteine in einem äußeren Magnetfeld (z.B. dem Erdmagnetfeld) abkühlen.


\section{Magnetisches Moment}
Das magnetische Moment $\vec{m}$  wird über das Integral
\begin{equation}
\vec{m}  = \int_V \vec{M}(\vec{x}) dV
\end{equation}
berechnet.
Ist die Magnetisierung homogen, so ist das magnetische Moment
\begin{equation}
\vec{m}  = \vec{M} \,V.
\end{equation}

Auf einen magnetisierten Körper in einem Magnetfeld wirkt eine Kraft, die zu dem Drehmoment $\vec{D} = \vec{m} \times \vec{B}$ führt. Die Einheit dieses Drehmoments ist [$\vec{D}$]= 1 Nm.

\section{Das Magnetfeld der Erde}
Um Anomalien des Magnetfeldes erkennen zu können, muss man das Magnetfeld der Erde gut kennen.\\
Das Erdmagnetfeld entspricht in erster Näherung einem Dipol mit einem magnetischen Moment von $\SI{8 e22}{Am^2}$, der seinen Ursprung im Erdkern hat. Es gibt eine Abweichung der Achse dieses Dipols und der 
Rotationsachse der Erde, dies wird als \textbf{Deklination} $D$ bezeichnet. 
Der Einfallswinkel der magnetischen Feldlinien wird \textbf{Inklination} $I$ genannt. \\
Betrachtet man nur das Magnetfeld, dass im Erdkern entsteht, so weicht es trotzdem von dem eines Dipols ab. \\ 94 \% des Erdmagnetfeld entsteht im Erdinneren. Hier sind die Quellen neben dem Kern auch magnetisierte Gesteine 

\section{Messinstrumente}

\subsection{Torsions-Magnetometer}

Beim Torsions-Magnetometer ist ein kleiner Stabmagnet zwischen zwei Torsionsfäden aus Metall drehbar aufgehängt. Ein äußeres Magnetfeld dreht diesen Stabmagnet, wodurch die Fäden torsioniert werden. Durch mechanisch verursachte Torsion der Fäden in die andere Richtung, kann der Stabmagnet wieder in seine Ruhelage zurückgebracht werden. Der dazu benötigte Torsionswinkel $\alpha$ kann danach abgelesen werden. Das Drehmoment $|\vec{D}|$ kann nun mit der Torsionskonstante $\tau$ und der bekannten Ruhelage $\alpha_0$ nach Gleichung
\begin{equation}
 |\vec{D}|=\tau(\alpha-\alpha_0)
\end{equation}
berechnet werden. Für die Komponente der magnetischen Flussdichte $\vec{B}$, die senkrecht zu $\vec{m}$ und der Torsionsachse steht, folgt der Zusammenhang
\begin{equation}
 B=\frac{\tau(\alpha-\alpha_0)}{|\vec{m}|} \fullstop
\end{equation}
Im Versuch wird ein Gfz-Instrument verwendet, das so ausgerichtet ist, dass die Vertikalkomponente $Z$ der magnetischen Flussdichte gemessen wird. Es wird dazu genutzt, die Vertikalkomponente an einem Ort im Laufe des Messtags zu messen, um zeitliche Variationen zu bemerken. Dies ist nötig, um die räumlichen Anomalien von diesen trennen zu können.
% noch nichts zur Kalibrierung geschrieben. Vllt bei der Versuchsdurchführung???
% Temperaturabhängigkeit

\subsection{Protonen-Präzessions-Magnetometer}

Mit Protonen-Präzessions-Magnetometern (kurz: Protonenmagnetometer) wird die Totalintensität $T$ gemessen. In die Messungen gehen nur atomphysikalische Konstanten und eine Zeitmessung ein, sodass die Messung unabhängig von Temperatureffekten ist.

In einem Behälter befindet sich Petroleum oder eine andere wasserstoffreiche Flüssigkeit. Dieser ist umgeben von einer Spule. Es muss eine wasserstoffreiche Flüssigkeit verwendet werden, weil die Präzession der Spins der Protonen in den Wasserstoff-Kernen beobachtet wird. In die Spulen wird ein Gleichstrom eingespeist , der ein Magnetfeld erzeugt. Die Spins der Protonen richten sich entlang des resultierenden Magnetfelds aus, das sich aus dem Erdmagnetfeld und dem durch die Spulen erzeugten Magnetfeld zusammen setzt. Das Feld der Spulen ist sehr viel größer als das Erdmagnetfeld. Die Ausrichtung der Spins ist also nahezu entlang des angelegten Feldes. Nach ein paar Sekunden wird das künstliche Feld schnell ausgeschaltet und die Spins präzedieren nun um die Richtung des Erdmagnetfelds, wodurch ein makroskopisches Wechselfeld entsteht. Dieses induziert eine Wechselspannung in der Spule, an der die Präzessionsfrequenz (Lamorfrequenz) $\q{\omega}{L}$ gemessen werden kann. Diese ist proportional zur magnetischen Flussdichte
\begin{equation}
 \q{\omega}{L}=\frac{\q{g}{I}\q{\mu}{K}}{\hbar}|\vec{B}|=\gamma |\vec{B}| \fullstop
\end{equation}
Der Kern-g-Faktor $\q{g}{I}$ des Protons ist
\begin{equation}
 \q{g}{I}=5,585694702\comma
\end{equation}
das Kernmagneton
\begin{equation}
 \q{\mu}{K}=\e{5,050824\cdot 10^{-7}}{Am^2}
\end{equation}
und die Plancksche Konstante
\begin{equation}
 \hbar=\frac{h}{2\pi}=\e{1,0545887\cdot 10^{-34}}{Js} \fullstop
\end{equation}
Daraus ergibt sich das konstante gyromagnetische Verhältnis der Protonen
\begin{equation}
 \gamma=\frac{\q{g}{I}\q{\mu}{K}}{\hbar}=\eb{2,67522205}{1}{Ts} \fullstop
\end{equation}
Aus der Lamorfrequenz kann also die magnetische Flussdichte bestimmt werden, die vom Protonenmagnometer direkt in nT ausgegeben wird.

Die Messgenauigkeit liegt im Bereich von $\e{1}{nt}$ bis $\pm\e{0,1}{nT}$. Protonenmagnetometer sind bei starken Wechselfeldern und großen räumlichen Gradienten des Magnetfeldes ungeeignet.