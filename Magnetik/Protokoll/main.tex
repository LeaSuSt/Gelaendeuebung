\documentclass{include/protokollclass}
% Main File - Based on protokollclass.cls
% Comments are mostly in English (and some in German, concerning the Praktikum)
% ------------------------------------------------------------------------------
% Further files in folder:
%  - include/cmds.tex (for macros and additional commands)
%  - include/kitlogo.pdf (for titlepage)
%  - lit.bib (bibtex bibliography database)
%  - include/titlepage.tex (for layout of titelpage)
% ------------------------------------------------------------------------------
% Useful Supplied Packages:
% amsmath, amssymb, mathtools, bbm, upgreek, nicefrac,
% siunitx, varioref, booktabs, graphicx, tikz, multicol

\usepackage{rotating}
\usepackage{icomma}
\usepackage{subfig}
\usepackage{pdfpages}
\usepackage[onehalfspacing]{setspace}



%% ---------------------------------------------
%% |    Informationen über dieses Protokoll    |
%% ---------------------------------------------
\newcommand{\praktikum}{P1}                % P1 oder P2
\newcommand{\semester}{WS16/17}            % z.B. "WS14/15" oder "SS15"

\newcommand{\wochentag}{Di}                % Mo, Di, Mi oder Do
\newcommand{\gruppennr}{04}                % Zweistellige Gruppennummer

\newcommand{\nachnamea}{Friedrich}             % Nachname des ersten Praktikanten
\newcommand{\vornamea}{Tabea}               % Vorname des ersten Praktikanten
\newcommand{\nachnameb}{Stockmeier}              % Nachname des zweiten Praktikanten
\newcommand{\vornameb}{Lea}              % Vorname des zweiten Praktikanten

\newcommand{\emailadressen}{lea.stockmeier@web.de, tabea.friedrich@t-online.de}
% optionale Angabe von Emailadresse(n) für den Kontakt mit dem Betreuer

\newcommand{\versuch}{Magnetik} % Name des Versuchs
\newcommand{\versuchsnr}{80}               % bitte die korrekte Nummer dem 
                                           % Arbeitsplatz am Versuchstag 
                                           % entnehmen
\newcommand{\fehlerrechnung}{Nein}         % Ob Fehlerrechnung im Versuch 
                                           % durchgeführt wurde oder nicht

\newcommand{\betreuer}{M. Mustermann}      % Name des zuständigen Betreuers
\newcommand{\durchgefuehrt}{01.09.16}      % Datum, an dem der Versuch 
                                           % durchgeführt wurde





%% --------------------------------------
%% |    Settings for Word Separation    |
%% --------------------------------------
% Help for separation:
% In German package the following hints are additionally available:
% "- = Additional separation
% "| = Suppress ligation and possible separation (e.g. Schaf"|fell)
% "~ = Hyphenation without separation (e.g. bergauf und "~ab)
% "= = Hyphenation with separation before and after
% "" = Separation without a hyphenation (e.g. und/""oder)

% Describe separation hints here:
\hyphenation
{
    über-nom-me-nen an-ge-ge-be-nen
    %Pro-to-koll-in-stan-zen
    %Ma-na-ge-ment  Netz-werk-ele-men-ten
    %Netz-werk Netz-werk-re-ser-vie-rung
    %Netz-werk-adap-ter Fein-ju-stier-ung
    %Da-ten-strom-spe-zi-fi-ka-tion Pa-ket-rumpf
    %Kon-troll-in-stanz
}





% um die Titelseite per PDF-reader auszufüllen. Vorgefertigte Daten
% können in Datei 'data.tex' modifiziert werden.
%\setboolean{forminput}{true}
% um die Anmerkungen zu den Textfeldern anzeigen zu lassen
%\setboolean{showannotations}{true}
% Erneuern der Seitenzahl in jedem Kapitel
%\setboolean{chapResetPageNumb}{true}
% Einbinden der Kapitelnummer in der Seitenzahl
%\setboolean{chapWiseNumb}{true}
% english or ngerman (new german für neue deutsche Rechtschreibung statt german)
\SelectLanguage{ngerman}

\title{Geophysikalische Geländeübungen \\ SS 2018 \\ Magnetik}
\subtitle{Messgebiet A59/1 (Riedheim)}
\author{\\ Svenja Müller \\ mueller-svenja@gmx.net
\\ \\und\\ \\
Lea Stockmeier \\ lea.stockmeier@web.de \\ \\ \\
Betreuer: Vorname1 Nachname1 und Vorname2 Nachname2}
\date{\vfill\vfill\vfill \today}




%% -----------------------
%% |    Main Document    |
%% -----------------------
\begin{document}
    % Titlepage und ToC
    \FrontMatter

    \maketitle

    \begingroup \let\clearpage\relax    % in order to avoid listoffigures and
    \tableofcontents                    % listoftables on new pages
    \listoffigures
    \listoftables
    \endgroup
    %\cleardoublepage



    % Contents
    \MainMatter
    
    
    \emptychapter[1]{Messprotokoll 1}{} % usage: \emptychapter[page displayed 
                                        %        in toc]{name of the chapter}
    \pseudochapter[3]{Messprotokoll 2}  % usage: \pseudochapter[number of pages 
                                        %        added]{name of the chapter}
                                        
    \chapter{Einleitung}
    Bei der geophysikalischen Geländeübung 2018 führten wir am ersten Messtag, den 22.05., die Magnetik-Messungen durch. Das folgende Protokoll beschreibt die Theorie, Versuchsdurchführung, Auswertung und Fehlerdiskussion zu diesem Versuch.

Bei den Messungen und im Protokoll verfolgten wir folgende Fragestellungen: Die erste große Fragestellung ist, ob mit der Magnetik der Gang lokalisiert werden kann. Bei der Kartierung stellten wir uns die Frage, ob wir den Gang so gut lokalisieren können, dass wir uns daran die Lage der weiteren Profile überlegen können.
Bei den Profilen stellten wir uns dann die Frage, ob diese wirklich senkrecht zum Gang angelegt wurden. Dazu dient vor allem ein Profil, das extra schräg zum Gang gewählt wurde, um zu sehen, ob überhaupt ein Unterschied festgestellt werden kann. Die senkrechten Profile sollen zeigen, ob ein gemeinsames, alle Verläufe der Totalintensitäten erklärendes Modell gefunden werden kann. Die Messungen mit dem Fluxgate verfolgen die Fragestellung, ob dabei sinnvoll das Zweikreisverfahren angewendet werden kann. Die Vermessung eines vorbeifahrenden Traktors und der Umgebung der Basisstation dienten zur Abschätzung der während der Messung durch äußere Einflüsse auftretenden Fehler.
    
    \chapter{Theoretische Grundlagen}
    \section{kjgkjgr}
    
    \chapter{Auswertung}
%     \input{./chap/chapter1.tex} %\cleardoublepage

    % appendix for more or less interesting calculations
    \Appendix
    \chapter*{\appendixname} \addcontentsline{toc}{chapter}{\appendixname}
    % to make the appendix appear in ToC without number. \appendixname = 
    % Appendix or Anhang (depending on chosen language)
    \section{Messprotokolle}

\begin{figure}[h!]
 \centering
 \includegraphics[width=0.8\textwidth]{fig/Messprotokolle/Kalibrierung.png}
 \caption{Messprotokoll zur Kalibrierungsmessung}
 \label{fig:MPKalibrierung}
\end{figure}

\begin{figure}[h!]
 \centering
 \includegraphics[width=\textwidth]{fig/Messprotokolle/EinflussHuette.png}
 \caption{Messprotokoll zum Profil zur Untersuchung der Einflüsse äußerer Störfaktoren auf die Basismessung}
 \label{fig:MPHuette}
\end{figure}

% \begin{figure}[h!]
%  \centering
%  \includegraphics[width=\textwidth]{fig/Messprotokolle/}
%  \caption{}
%  \label{fig:}
% \end{figure} %\cleardoublepage



    % Bibliography
    \TheBibliography

    % BIBTEX
    % use if you want citations to appear even if they are not referenced to: 
    % \nocite{*} or maybe \nocite{Kon64,And59} for specific entries
    %\nocite{*}
    \bibliographystyle{babalpha}
    \bibliography{lit.bib}

    % THEBIBLIOGRAPHY
    %\begin{thebibliography}{000}
    %    \bibitem{ident}Entry into Bibliography.
    %\end{thebibliography}
\end{document}
