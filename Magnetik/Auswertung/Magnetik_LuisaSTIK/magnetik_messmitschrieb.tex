\documentclass[a4paper,10pt]{article}

% Kommentare: Alles hinter dem Zeichen "%" wird ignoriert.

% Zus"atzliche Pakete einbinden:
\usepackage[utf8]{inputenc} % So können Umlaute auch direkt eingegeben werden.
\usepackage{ngerman}  % Deutsche Sprachumgebung (neue Rechtschreibung)
\usepackage{amsmath}  % erweiterter Formelsatz
\usepackage{amssymb}  % weitere mathematische Symbole
\usepackage{amsfonts} % sch"one Fonts besonders f"ur Formeln
\usepackage{graphicx} % Bilder einbinden
\usepackage{hyperref} % Dokumentnavigation
\usepackage{booktabs} % für schönere Linien in Tabellen
%\usepackage{fullpage} % minimale Seitenr"ander, ist allerdings nicht immer verfügbar, 
% daher die alternative:
\usepackage[margin=1in,includefoot,footskip=30pt]{geometry}

\usepackage[load-configurations=abbreviations]
{siunitx}                                       % package for units. Use 
                                                % \si{\ampere} or 
                                                % \SI{0.3}{\angstrom}
% Keine Einr"uckung bei neuem Paragraphen 
\parindent 0pt
\usepackage{here}
\usepackage{subfigure}
\usepackage{siunitx}





\usepackage{upgreek}
\usepackage{footmisc}
\setlength{\footnotemargin}{0em}
\numberwithin{equation}{section}
%% Definitionen f"ur das Titelblatt

\date{22.05.2018}

\begin{document}


\section{Torsionsmagnetometer: Basiststation (Pflock 103)}


\subsection{Kalibrierungsmessung}
durchgeführt von Svenja beim Ablesen am Magnetometer und Kati an der Stromquelle \\
Fehlerquellen Kalibrierungsmessung: Horizontierung, Hütte mit unbekanntem Inhalt \\
Beginn der Messung: etwa 12 Uhr \\

\subsection{Tagesgangmessung}
Fehlerquellen: Hütte mit unbekanntem Inhalt; Traktor parkte in geschätzt 80 Metern Entfernung von 13:30 bis 14:00 Uhr \\
13:20 Uhr: erneute Horizontrierung \\
13:30 Uhr: stand kurz in der Sonne $\rightarrow$ Verschiebung des Schirms \\
Messung durchgeführt von Svenja


\section{Gradiometermessung}
Koordinatensystem: s. png \\
Beginn: 13 Uhr; Rebekka startet bei (0|0), Ende: 13:20 Uhr \\
Wechsel: Lea startet um 13:20 Uhr bei (0|17) (ein Testlauf, dieser wurde gelöscht), ab und zu zu schnell und mal zu langsam gelaufen; Ende: etwa 14 Uhr \\
Messung in x-Richtung in 1-Meter-Abständen von (x|0) bis (x|29) immer \\
Samplingrate in x-Richtung: 8 Messungen pro Meter $\rightarrow$ 240 Messungen pro Reihe $\rightarrow$ 720 Messungen  insgesamt \\
Begründung für gewähltes Messprofil: geologische Karte $\rightarrow$ Basaltgang läuft in N-S-Richtung unter der Wiese durch $rightarrow$ Luisa und Rebekka sind mit Gradiometer in den drei gemähten Reihen durchgelaufen und haben das Gebiet abgeschätzt, in dem das Gradiometer angeschlagen ist \\
Messdurchführung: zwei Maßbänder in 1\,m Abstand; mit Messgerät an einem orientierung und dort entlang laufen \\
Auffälligkeiten bei der Messung: \\
(x|etwa 20) Traktor fuhr während der Messung vorbei \\
(x|27) Busch im Weg $\rightarrow$ Umgehen \\


\end{document}