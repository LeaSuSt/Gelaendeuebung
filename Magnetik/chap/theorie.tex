\section{Einführung}

Mit diesem Verfahren kann die Magnetisierung und magnetische Suszebilität von Objekten im Untergrund untersucht werden, sowie Variationen des Erdmagnetfelds. Es kann sowohl die räumliche als auch die zeitliche Variation 
betrachtet werden.\\

Das gesamte gemessene Magnetfeld setzt sich aus dem Hintergrundfeld und Anomalien zusammen. Diese Anomalien 
 werden durch magnetisierbare und remanent magnetisierbare Gesteine hervorgerufen werden. 



Die Messgröße die in diesem Versuch gemessen wird ist die \textbf{magnetische Flussdichte} $\vec{B}$, sie gibt die Stärke des Magnetfeldes an. \\
Eine andere wichtige Größe ist \textbf{Magnetisierung} $\vec{M}$ eines Materials, sie hängt sowohl von der Stärke des Magnetfelds als auch den Materialeigenschaften ab.\\

\section{induzierte und remantente Magnetisierung}
Gesteine oder sonstige Störkörper könne remanent oder induziert magnetisiert sein. Induzierte Magnetisierung ist Magnetisierung, die nur solange ein äußeres Magnetfeld angelegt ist, vorhanden ist. Remanent magnetisierte 
Materialien sind dauerhaft magnetisch, auch ohne äußeres Magnetfeld.\\
Dia- und paramagnetische Materialien weißen nur induzierte Magnetisierung auf. Ferro-, ferri- und antiferromagnetische Materialien können auch remanent magnetisiert sein. \\
Daneben gibt es noch die thermoremanente Magnetisierung, hierbei entsteht die Magnetisierung während Gesteine in einem äußeren Magnetfeld (z.B. dem Erdmagnetfeld) abkühlen.


\section{Magnetisches Moment $\vec{m}$}
Das magnetische Moment [$\vec{m}$]  wird über das Integral
\begin{equation}
\vec{m}  = \int_V \vec{M}(\vec{x}) dV
\end{equation}
berechnet.
Ist die Magnetisierung homogen, so ist das magnetische Moment
\begin{equation}
\vec{m}  = \vec{M} \,V.
\end{equation}

Auf einen magnetisierten Körper in einem Magnetfeld wirkt eine Kraft, die zu dem Drehmoment $\vec{D} = \vec{m} \times \vec{B}$ führt. Die Einheit dieses Drehmoments ist [$\vec{D}$]= 1 Nm.

\section{Das Magnetfeld der Erde}
Um Anomalien den Magnetfeldes erkennen zu können, muss man das Magnetfeld der Erde gut kennen.\\
Das Erdmagnetfeld entspricht in erster Näherung einem Dipol mit einem magnetischen Moment von $\SI{8 e22}{Am^2}$, der seinen Uhrsprung im Erdkern hat. Es gibt eine Abweichung der Achse dieses Dipols und der 
Rotationsachse der Erde, dies wird als \textbf{Deklination} $D$ bezeichnet. 
Der Einfallswinkel der magnetischen Feldlinien wird \textbf{Inklination} $I$ genannt. \\
Betrachtet man nur das Magnetfeld, dass im Erdkern entsteht, so weicht es trotzdem von dem eines Dipols ab. \\ 94 \% des Erdmagnetfeld entsteht im Erdinneren. Hier sind die Quellen neben dem Kern auch magnetisierte Gesteine 




