%Zusammenfassung Seismik 
Die Messung auf dem Profil S11-S12, die mit S.I.S.Sy durchgeführt wurde, liefert sehr gute Ergebnisse.
Unsere Fragestellung war in erster Linie der Vergleich dieser Messmethode mit der Geoelektrik. 
In der Geoelektrik hat man viel Interpretationsfreiraum, was die Anzahl und die Tiefe von Schichten angeht.
Die Seismik liefert eindeutigere Ergebnisse. So sind wir uns sehr sicher, das wir genau eine Schichtgrenze in einer Tiefe von $d_+= \SI{4,57 \pm 1,50}{m}$
bzw. $d_-= \SI{3,35 \pm 1,09}{m}$ haben. Die Schicht ist geneigt, weshalb es auch zwei Schichttiefen gibt. Eine am Punkt des Hinschusses und eine am Punkt des Rückschusses. In der Geoelektrik wurde in einem der berechneten Modelle ebenfalls eine Schichtgrenze in etwa dieser Tiefe gefunden. 
Ändern sich die seismischen und geoelektrischen Eigenschaften gleichzeitig, kann es sich um die gleiche Schicht handeln.

Optimal wäre es für die Messung gewesen, wenn wir ein ebenes Profil verwendet hätten. Wir haben deswegen ein Profil mit möglichst wenig Topographie gesucht.
Eine kleinere Erhebung haben wir aber leider trotzdem auf dem Messprofil. Dieser Hügel ist vermutlich als Kurve im Laufzeitdiagramm zu erkennen.
Trotzdem können wir sagen, dass sich die Seismik gut zum Untersuchen des Untergrunds auf Profil S11-S12 eignet.

Anders verhält es sich beim Versuch, den Basaltgang mit der Hammerschlag-Methode nachzuweisen. 
Eine Messung wurde auf Profil S21-S22 direkt über dem Basaltgang durchgeführt. Die Vergleichsmessung (Profil S31-S32) parallel dazu und weit genug weg vom Basaltgang diente dazu, den Einfluss des Basalts besser deuten zu können.
Für beide Messungen wurden Schichtgrenzen in etwa den gleichen Tiefen bestimmt. Dies bedeutet, dass wir den Basaltgang mit der Messung nicht nachweisen konnten. 
Die berechneten Schichtgrenzen sind wohl die des Untergrunds um den Basaltgang.

Einen Unterschied gibt es jedoch zwischen den beiden Laufzeitdiagrammen. Die Messwerte der Vergleichsmessung bilden eine relativ schöne, gleichmäßige Kurve. 
Im Laufzeitdiagramm der Messung über dem Basaltgang gibt es viele kleine Ausreißer der Messwerte nach oben oder unten. Dies lässt sich vermutlich damit erklären,
dass die seismischen Wellen teilweise an der Oberkante des Basaltgangs unregelmäßig gestreut und reflektiert wurden.

Der Basaltgang konnte mit der Seismik also nicht lokalisiert werden. Von allen Messmethoden lieferte sie dadurch die schlechtesten Ergebnisse über dem Basaltgang.