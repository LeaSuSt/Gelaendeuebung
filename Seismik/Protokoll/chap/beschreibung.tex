Als Gruppe wollten wir sehr gerne die Oberkante des Basaltgangs mit der Refraktionsseismik detektieren. Von den Betreuern ??? wussten wir aber, dass Gruppen in den Vorjahren Schwierigkeiten damit hatten. Wir entschieden uns also, zunächst auf einer anderen Wiese ein Profil anzulegen und erst dann beim Basaltgang zu messen. Die Lage der vermessenen Profile ist in Abbildung ??? zu sehen und das Messprotokoll befindet sich in Abbildung \ref{fig:messprotokoll}

\section{Profil S11-S12}

Dieses Profil liegt auf einer anderen Wiese als der mit Basaltgang. Bei der Refraktionsseismik wird die Annahme getroffen, dass ebene, homogene Schichten im Untergrund vorliegen. Bei geradem Boden ist die Wahrscheinlichkeit größer, dass dies näherungsweise der Fall ist. Wir versuchten also eine möglichst flache Stelle zu finden, was sich aufgrund der ausgeprägten Topographie jedoch als sehr schwierig herausstellte. Man konnte wegen eines Hügels vom einen Ende des Profils das andere nicht sehen. In Abbildung ??? ist die genaue Lage der 72 verwendeten Geophone eingezeichnet.

\subsection{Hammerschlag}

Von den Betreuern wurde uns empfohlen, beim Hammerschlag einen Hin-, Rück- und Mittelschuss durchzuführen, weil die Gesamtlänge von 138\,m durch Hammerschläge vermutlich nicht erreicht werden kann. Außerdem stapelten wir bei der Messung mit Hammerschlag jeweils 5 Messungen, um ein besseres Signal-Rausch-Verhältnis zu erreichen. Der Punkt des Mittelschuss wurde S13 genannt und befindet sich 49\,m vom Punkt des Hinschusses S11 entfernt, wie in Abbildung ??? eingezeichnet.

\subsection{Verpuffungsquelle S.I.S.Sy}

Bei der Messung mit der Verpuffungsquelle S.I.S.Sy als  Signalquelle musste nur ein Hin- und ein Rückschuss durchgeführt werden, da sie eine höhere Energie als ein Hammerschlag aufweist. Die Löcher wurden genau an den Ende des Profils gebohrt, was zu einer Gesamtlänge dieser Auslage von 139\,m ergab. Beim Bohren ergab sich die Schwierigkeit, dass wir das erste Loch mit einem Knick bohrten und so S.I.S.Sy nicht hineinpasste. Außerdem stießen wir in einer Tiefe von ca. 60\,cm auf eine Schicht mit sehr vielen Steinen.

\section{Profile beim Basaltgang}

Nun untersuchten wir noch den Basaltgang mit der Refraktionsseismik. Dazu legten wir ein Profil über dem Basaltgang und eines weit neben ihm an. Das zweite Profil soll zum Vergleich dienen, um zu sehen, ob der Gang mit der Seismik überhaupt detektiert werden kann. Bei beiden Profilen wurden 24 Geophone auf einer Länge von 42\,m verwendet (Siehe Abbildung ???). Wegen der geringen Länge wurden nur Hin- und Rückschüsse mit Hammerschlag durchgeführt. Es wurden wieder 5 Seismogramme gestapelt.

\subsection{Profil S21-S22}

Dieses Profil legten wir so an, dass es möglichst parallel zum und genau über dem Basaltgang war. Dazu nutzten wir unser Wissen über die Lage des Basaltgangs aus der Magnetik-Kartierung (Wie aufs Protokoll verweisen???) vom Vortag.

\subsection{Profil S31-S32}

Per Augenmaß parallel zu Profil S21-S22 legten wir dieses Profil an, das möglichst weit weg vom Basaltgang sein sollte, um auschließen zu können, dass er noch einen Einfluss auf die Messergebisse hat.