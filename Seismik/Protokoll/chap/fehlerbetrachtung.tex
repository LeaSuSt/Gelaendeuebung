%Fehlerbetrachtung
Bei der Berechnung der Fehler in der Auswerung wurde eigentlich nur der Fehler berücksichtigt, der bei legen der Ausgleichsgeraden gemacht wird. Aber auch während der Messung gibt es viele Ungenauigkeiten, die zu Fehlern führen können. \\
Eine Fehlerquelle ist das Auslegen der Geophone. Diese werden in einem bestimmten Abstand entlang eines Messbands gesteckt. Da das Band nicht gerade auf dem Boden Aufliegt, sondern durch Gras geht sind schon hier die Abstandsangaben ungenau. Dadurch ist auch der Fehler auf die Skala des Massbands zu vernachlässigen. Des weiteren können, wegen dem Untergrund, die Geophone auch nicht immer an dem richtigen Ort am Massband gesteckt werden. Die Geophone müssen, um optimale Werte zu bekommen, horizontal zum Boden gesteckt sein. Auch dies war oft nicht möglich.\\
Wenn sich Personen neben den Geophonen bewegen, kann das registriert werden. Daher haben sicher auch vorbeifahrende Autos usw einen großen Einfluss. Wir haben jedoch darauf geachtet, das kein Auto in der Nähe der Messung war. Auch die Bewegung der Bäume duch wird kann zu Ändereungen der Messwerte führen. Das sind Fehler auf die man wenig bis keinen Einfluss hat. 
Bei der Hammerschlag-Methode können veränderte Werte entstehen, wenn man mit dem Hammer nicht die Mitte der Metallscheibe trifft, oder unterschiedliche Personen schlagen. 
Wir haben uns bei der Messung bemüht, diese Fehlerquellen gering zu halten. Aber einen nicht genau berechenbaren Einfluss haben sie trotzdem.