Der Versuch Refraktionsseismik der geophysikalischen Geländeübung wurde am zweiten Tag der Messwoche, dem 23.05., durchgeführt.

Die Fragestellungen unterschieden sich dabei bei den drei verschiedenen Profilen, auf denen Messungen durchgeführt wurden.

Ein Profil längs des Basaltgangs soll die Frage beantworten, ob dieser mit der Refraktionsseismik lokalisiert werden kann. Wenn dies der Fall ist, ist sollte dadurch die Tiefe der Oberkante bestimmt werden. Ein Vergleichsprofil außerhalb des Einflussbereichs des Gangs soll eindeutig zeigen, ob der Gang einen Einfluss auf das Laufzeitdiagramm hat. 

Ein anderes Profil wurde auf einer etwas abgelegenen Wiese ausgelegt. Es dient dazu, neben der Hammerschlagseismik auch noch eine Messung mit einer Verpuffungsquelle durchführen zu können. Die Fragestellungen dabei sind, ob Schichten unterschiedlicher seismischer Geschwindigkeiten gefunden werden können und in welcher Tiefe diese liegen.