% Gliederung noch anders!!!???
\section{Laufzeitdiagramm}

% Verweise lieber woanders?
Am Abend nach dem Versuch erhielten wir ausgedruckt alle Seismogramme. Die mit Hammerschlag aufgenommenen direkt gestapelt. Daraus pickten wir die Ersteinsätze. Wie genau gepickt wurde, ist in den Abbildungen ??? bis ??? zu sehen. Die Seismogramme des Profils S11-S12 mit Hammerschlag wurden nicht gepickt, da diese Messung in diesem Protokoll nicht weiter ausgewertet wird. Die zugehörigen Laufzeitdiagramme der Hin- und Rückschüsse befinden sich in Abbildung ???, ??? und ???.

\section{Schichtbestimmung}

\subsection{Profil S31-S32}

Das Laufzeitdiagramm mit eingetragenem Hin- und Rückschuss dieses Profils ist achsensymmetrisch um $x=20$. Daraus lässt sich schließen, dass keine geneigten Schichtgrenzen vorliegen. Aus den 


% Verfahren der Verzögerungszeiten wenn sinnvoll???

\subsection{Profil S21-S22 }

Dieses Profil liegt über dem Basaltgang. Vergleicht man das Diagramm der Messung über und neben dem Basaltgang, stellt man fest, dass die Werte des Profils S31-S32 wesentlich weniger Schwankungen haben als die des Profils S21-S22.
Wir vermuten, das diese Schwankungen von Basaltgang verursacht werden. Bei der Refraktionsseismik geht man von unendlichen geraden Platten aus. Wie wir in der Geoelektrik herausgefunden haben ist der 



