%Fehlerbetrachtung Geoelektrik
Unsere Messwerte können durch viele Fehler bei der Durchführung der Messung und Auswertung beinflusst werden. Dabei überwiegen die systematischen Fehler. So wird z.B von ebenen Schichten ausgegangen. Dies ist mit hoher wahrscheinlichkeit nicht gegeben. 
Die Messung wird auch durch viele Umwelteinflüsse beeinflusst. Dazu zählen künstliche Ströme an der Erdoberfläche oder Bäume die durch Wasserspeicher in ihren Würzeln die elektrische Leitfähigkeit lokal erhöhen. Entlang des Profils an dem die Wennerkartierung und Tomagraphie durchgeführt wurde stehen sehr viele Bäume. Diese Messungen wurden also wahrscheinlich stark von diesen lokalen Wasserspeichern beeinflusst. In unseren Messwerten sind aber keine Anomalien zu erkennen, die darauf zurück zu führen wären.\\
Eine weitere Fehlerquelle ist das Stecken der Elektroden. Beim Stecken der Elektroden hat man sich an einem Massband orientiert. Dieses war aber über teilweise ungemähtes Gras gelegt, was sicher einen Fehler von durchschnittlich  $\SI{\pm 0,2}{m}$ ausmacht. Daher kann der Fehler auf die Skala des Messbands vernachlässigt werden.  
Mit diesem Fehler und Gaußscher Fehlerfortpflanzung 
\begin{equation}
\sigma_y = \sqrt{\sum_i \left( \frac{\partial y}{\partial x_i}\cdot \sigma_{x_i}\right)^2}
\end{equation}
kann nun ein Fehler auf den Geometriefaktor berechnet werden. Der Geometriefaktor wird berechnet mit 

$$F = \frac{2 \pi}{\frac{1}{r_{\mathrm{AM}}} - \frac{1}{r_{\mathrm{MB}}} + \frac{1}{r_{\mathrm{NB}}} - \frac{1}{r_{\mathrm{AN}}}}\,,$$
wobei $ r_\mathrm{AM} = r_\mathrm{NB} = \SI{5 \pm 0.2}{m}$  und $ r_{\mathrm{AN}} =  r_{\mathrm{MB}} = \SI{10 \pm 0.4}{m}$ ist.

Der Geometriefaktor ist $f= 31,4 \pm 2 $. 
Wir beechnen nur auf den größten Wert des spezifischen Widerstands einen Fehler, da dieser auch der größte Fehler ist. 

Die Berechnugn wurde mit Python durchgeführt, der spezifische Wiederstand
\begin{equation}
\rho = \frac{V}{I} \, F. 
\end{equation}


ist $\rho= \SI{33,9 \pm 2,14}{\Omega m}$.

Dieser Fehler ist relativ gering. Wir vermuten dass Fehler die z.B. durch die Annahme gerade, unendlich ausgedente Schichten im Untergrund gemacht werden wesentlich größer sind. Weshalb man diesen Fehler vernachlässigen kann.

Für die Sondierng waren die Abstände der Elektroden sehr groß gewählt, so das hier der Fehler durch das Massband vernachlässigt werden kann. Insgesammt finden wir es sehr schwer auf die Sondireung einen sinnvollen Fehler anzugeben, da wir nicht einemal mit sicherheit sagen können wie viele Schichten im Untergrund sind. Wir nehmen an, dass die Fehler durch die Annahme, dass im Untergrund gerade, unendlich ausgedente Schichten sind, so viel größer sind als alle Fehler, die sonst entstanden, dass es nicht viel Sinn macht, hier wirklich einen genauen Fehler auf unsere Werte anzugeben.

