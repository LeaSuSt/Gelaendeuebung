%Fehlerbetrachtung Geoelektrik
Unsere Messwerte können durch viele Fehler bei der Durchführung der Messung und Auswertung beinflusst werden. Dabei überwiegen die systematischen Fehler. So wird z.B von ebenen Schichten ausgegangen. Dies ist mit hoher wahrscheinlichkeit nicht gegeben. 
Die Messung wird auch durch viele Umwelteinflüsse beeinflusst. Dazu zählen künstliche Ströme an der Erdoberfläche oder Bäume die durch Wasserspeicher in ihren Würzeln die elektrische Leitfähigkeit lokal erhöhen. Entlang des Profils an dem die Wennerkartierung und Tomagraphie durchgeführt wurde stehen sehr viele Bäume. Diese Messungen wurden also wahrscheinlich stark von diesen lokalen Wasserspeichern beeinflusst. In unseren Messwerten sind aber keine Anomalien zu erkennen, die darauf zurück zu führen wären.\\
Eine weitere Fehlerquelle ist das Stecken der Elektroden. Beim Stecken der Elektroden hat man sich an einem Massband orientiert. Dieses war aber über teilweise ungemähtes Gras gelegt, was sicher einen Fehler von hast $\SI{\pm 0,2}{m}$ ausmacht. Daher kann der Fehler auf die Skala des Messbands vernachlässigt werden.  
Für die Sondierng waren die Abstände der Elektroden sehr groß gewählt, so das hier der Fehler durch das Massband vernachlässigt werden kann.

