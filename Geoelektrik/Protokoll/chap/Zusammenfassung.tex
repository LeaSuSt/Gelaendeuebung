%Zusammenfassung Geoelektrik
Alle drei der angewendeten Messmethoden haben gute Ergebnisse geliefert.

In den Messdaten der Tomographie und Wenner-Kartierung konnte eindeutig der Basaltgang erkannt werden. Die Geoelektrik eignet sich also insgesamt sehr gut zum Untersuchen des Basaltgangs,
anders als z.B die Seismik. Auch haben die Ergebnisse die Erkenntnisse der Magnetik ergänzt und bestätigt.
Was nicht gut gewählt wurde, ist die Auslage der Tomographie. Da wir auf einem relativ kurzen Profil mit kleinen Elektrodenabständen gemessen haben, reicht unsere Messung in nicht ganz 5\,m 
Tiefe. Wir haben die Wenner-Kartierung jedoch zuvor in einer Tiefe von 5\,m durchgeführt. Dadurch kann man diese beiden Methoden leider kaum vergleichen.

Die Schlumberger-Sondierung sollte mit der S.I.S.Sy-Seismik-Messung verglichen werden. Wir haben zwei Modelle für mögliche Schichten im Untergrund erstellt. Das zweite Modell,
bei dem von fünf Schichten ausgegangen wird,
hat eine Schichtgrenze, die wir auch mit der Seismik gesehen haben könnten. Bei diesem Modell gibt es in etwa 25\,m Tiefe eine Schichtgrenze, bei der die Leitfähigkeit sprunghaft zunimmt.
Hier vermuten wir, auf den Grundwasserspiegel gestoßen zu sein.
Da wir die Messung an einem Hang auf einem Hügel durchgeführt haben, erscheinen uns 25\,m realistisch als Grundwasserspiegel. 
Mit dieser Messmethode haben wir also auch auf dem zweiten Messgebiet gute Ergebnisse bekommen. 
Die Schlumberger-Sondierung eignet sich genauso wie die Seismik auf diesem Profil, um Messungen durchzuführen. 